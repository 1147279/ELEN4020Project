\documentclass[conference, 11pt]{IEEEtran}
\IEEEoverridecommandlockouts
% The preceding line is only needed to identify funding in the first footnote. If that is unneeded, please comment it out.
\usepackage{cite}
\usepackage{amsmath,amssymb,amsfonts}
\usepackage{algorithmic}
\usepackage{graphicx}
\usepackage{textcomp}
\usepackage{xcolor}
\def\BibTeX{{\rm B\kern-.05em{\sc i\kern-.025em b}\kern-.08em
    T\kern-.1667em\lower.7ex\hbox{E}\kern-.125emX}}
\usepackage{hyperref}
%Removing indentation.
\newlength\tindent
\setlength{\tindent}{\parindent}
\setlength{\parindent}{0pt}
\renewcommand{\indent}{\hspace*{\tindent}}
\usepackage{float}

\begin{document}

\title{Some title\\
{\footnotesize \textsuperscript{}ELEN4020}
}

\author{\IEEEauthorblockN{1\textsuperscript{st} Darren Blanckensee}
\IEEEauthorblockA{\textit{School of Electrical and Information Engineering} \\
\textit{University of the Witwatersrand}\\
Johannesburg, South Africa \\
1147279@students.wits.ac.za}
\and
\IEEEauthorblockN{2\textsuperscript{nd} Uyanda Mphunga}
\IEEEauthorblockA{\textit{School of Electrical and Information Engineering} \\
\textit{University of the Witwatersrand}\\
Johannesburg, South Africa \\
1168101@students.wits.ac.za}

\and
\IEEEauthorblockN{3\textsuperscript{nd} Ashraf Omar}
\IEEEauthorblockA{\textit{School of Electrical and Information Engineering} \\
	\textit{University of the Witwatersrand}\\
	Johannesburg, South Africa \\
	710435@students.wits.ac.za}

\and
\IEEEauthorblockN{4\textsuperscript{nd}Amprayil Joel Oommen}
\IEEEauthorblockA{\textit{School of Electrical and Information Engineering} \\
	\textit{University of the Witwatersrand}\\
	Johannesburg, South Africa \\
	843463@students.wits.ac.za}


}

\maketitle

\begin{abstract}
.
\end{abstract}



\section{Introduction}

\IEEEPARstart{T}{wo} parallel join algorithms were implemented and tested on two different parallel frameworks. The implemented algorithms are the join and hash joins. The hash join algorithm was implemented on MPI and the merge join algorithm was implemented on openMP. The purpose of the project are twofold, the primary task is to use parallelism to implement two different join algorithms and the secondary is to compare the algorithms' efficiency in joining. This project discusses existing solutions of other authors and the context in which this project was implemented. This report also contains sections on the Design and Implementation of the codes, Testing and Results, a SWAT analysis of the algorithms and possible Future Improvements in implementing the algorithms.


\section{Background}
	The join algorithms are designed to read data from  file of a specific format and to perform the required operations as per specific algorithm. The algorithms read data of the same file in order to be able to compare their performance. The algorithms are both implemented in C++. This allows for accuate comparisons of the algorithm performances and consequently prevents the possible time lags that may arise from the use of different programming languages. C++ was used instead of C due to the authors finding easier to use C++ under the given time constraints. Another reason that why C++ was found more favourable is that is offers more functionality than C and due to its Object Orientated Design~(OOD) capabilities and the also due to the fact that the authors are familiar with C++. These algorithms were implemented using parallelism.
	
	\subsection{Existing Solutions}
	Based on the literature, there are multiple join algorithms. The algorithms chosen by the authors are the hash algorithm and the merge algorithm, which form part of the equi-join family of join algorithms and are widely used~\cite{daniel}.
	
	\subsection{Literature Review}
	This section views some of the literature on the has and join algorithms.
	
	\subsubsection{Merge Join Review}
	References~\cite{daniel,wolf} take into account possible errors that may occur due to data skew for the merge join algorithm. According to~\cite{daniel}, data skew can result in some of the processors being over-utilised and others under-utilised. Reference~\cite{daniel} proposes the use of a divide-and-conquer technique as a solution to handle data skew. Reference~\cite{wolf} approaches the issue of data skew in the merge algorithm by adding an extra scheduling phase. The implemented merge join does not take into account the possibility of data skew and instead relies on openMP's computational ability to process certain chunks of data and ability to make numerous threads~\cite{omp}. This technique was chosen due to the resources available, namely a cluster, and time constraints.
	
	\subsubsection{Hash Join Review}
	The reviewed literature for the hash join is also concerned about data skew. Some of the said literature include references~\cite{dias,roy} and other documents. Reference~\cite{dias} addresses the issue of data skew by implementing multiple hash phases along with the use of a heuristic optimisation algorithm in order to isolate skew elements\cite{dias}. 
	
	
	\subsection{Merge Join}
	Merge join, as stated above, is also known as sort-merge join~\cite{goetz}. The basic idea of this algorithm is to sort data (in parallel in this context) according to the attributes that are going to be used in the joining process\cite{heap}. The implemented algorithm assumes that the sorting of the data has already taken place and proceeds to only perform the join algorithm under that assumption.
	
	\subsection{Hash Join}
	A hash join can be generalised into two stages~\cite{roy}, the split phase and the join phase~\cite{masaru}.
	
	
	
	\subsection{Assumptions}
	It is assumed that the data read has no errors and no error checks are provided.
	
	\subsection{Constraints}
	The experienced constraints included the allocated time to complete the project and the delays that were experienced due to technical issues, such as the malfunctioning of the cluster.
	
	\subsection{Success Criteria}
	Successful implementation of parallelism for the join algorithms.
	
	\subsection{Time Management and Work Division}
	---------------someone put a table here-------------------------
	
	
\section{Design and Implementation}
	\subsection{Merge Join}
	
	\subsection{Hash Join}

\section{Testing and Results}
	



\section{SWAT Analysis}
	\subsection{Strengths}
	
	\subsection{Weaknesses}
	
	\subsection{Advantages}
	
	\subsection{Threats}
	
	
\section{Future Improvements}

\section{Conclusion}

\bibliographystyle{IEEEtran}
\bibliography{dics}

\end{document}
